\section{Math}

\subsection{Geometric Series}
Geometric series for $q \neq 1$:
\[
\sum\limits_{k=1}^n q^k = \frac{1-q^{n+1}}{1-q}
\]
For $\lvert q \rvert < 1$ the corresponding infinite sum converges:
\[
\sum_{i = 1}^{\infty} q^k = \frac{1}{1 - q}
\]

\subsection{Picks Theorem}
For every polygon with integer-only coordinates with following holds
for the area $A$, amount of interior integer points $i$ and amount of
boundary points $b$:
\[
A = i + \frac{b}{2} - 1
\]
Calculation of border points:
\lstwot{geometry/picksTheorem.cc}
For polygon area see the geometry section.

\subsection{Eulers formular for planar graphs}
For every planar graph $G$ the following holds: $V - E + F = 2$

\subsection{Combinatorics}
%TODO(Alex): Basic combinatorics

\subsubsection{Binomial coefficient}
Number of possible sets with $k$ elements selected from $n$ elements. 
\begin{equation*}
  \binom{n}{k} = \binom{n}{n - k} = \frac{n!}{k!(n - k)!} =
  \frac{n}{k}\binom{n - 1}{k - 1} = \binom{n - 1}{k} + \binom{n - 1}{k
  - 1}
\end{equation*}
\lst{linear implementation}{math/binom.cc}

\subsubsection{Catalan numbers}
Use cases:
\begin{itemize}
\item The number of valid groupings of $n$ pairs of parentheses
\item The number of diagonal-avoiding paths on $n\times n$ grid
\item The number of ways to triangulate a regular polygon with $n + 2$
  sides
\item The number of rooted full binary trees with $n$ internal nodes
\item The number of rooted trees with $n$ edges
\item The number of ways to correctly parenthesize an ordered expression of $n
  + 1$ items with a binary operation
\end{itemize}
\begin{equation*}
  C_n = \frac{1}{n + 1}\binom{2n}{n} =
  \begin{cases}
    1 & n = 0 \lor n = 1\\
    \sum\limits_{k = 0}^{n - 1}C_kC_{n - 1 - k} & \text{otherwise}
  \end{cases}
\end{equation*}

\subsubsection{Euler numbers}

The number of permutations $1,\ldots, n$ with exactly $k$
non-decreasing Segments.
\begin{equation*}
  \left\langle
    \begin{matrix}
      n\\k
    \end{matrix}
\right\rangle = \sum\limits_{l = 0}^k(-1)^l \binom{n + 1}{l}(k + 1 - l)^n
  =
  \begin{cases}
    1 & k = 0 \lor k = n\\
    k \left\langle
      \begin{smallmatrix}
        n - 1\\k
      \end{smallmatrix}
\right\rangle + (n - k + 1) \left\langle
  \begin{smallmatrix}
    n - 1\\k - 1
  \end{smallmatrix}
\right\rangle & \text{otherwise}
  \end{cases}
\end{equation*}

\subsubsection{Stirling numbers of the first kind}
Number of permutations $1, \dots, n$ with exactly $k$ cycles.
\begin{equation*}
  \begin{bmatrix}
    n\\k
  \end{bmatrix} =
  \begin{cases}
    0 & k = 0\\
    1 & n = n\\
    \left[
      \begin{smallmatrix}
        n - 1\\k - 1
      \end{smallmatrix}
\right] + (n - 1) \left[
  \begin{smallmatrix}
    n-1\\k
  \end{smallmatrix}
\right] & \text{otherwise}
  \end{cases}
\end{equation*}

\subsubsection{Stirling Numbers of the second kind}

Number of partitions of $n$ elements into $k$ sets.
\begin{equation*}
  \begin{Bmatrix}
    n\\k
  \end{Bmatrix} =
  \begin{cases}
    1 & k = 1 \lor k = n\\
    \left\{
      \begin{smallmatrix}
        n - 1\\k - 1
      \end{smallmatrix}
\right\} + k \left\{
  \begin{smallmatrix}
    n - 1\\k
  \end{smallmatrix}
\right\} & \text{otherwise}
  \end{cases}
\end{equation*}



\subsubsection{Derangements}
Amount of permutations of a set with $n$ elements such that no element
is at its starting position.
\begin{equation*}
  \text{der}(n)
  \begin{cases}
    1 & n = 0\\
    0 & n = 1\\
    (n - 1)(\text{der}(n - 1) + \text{der}(n - 2)) & \text{otherwise}
  \end{cases}
\end{equation*}

\subsubsection{Hypergeometric distribution}
Set with $N$ elements of which $K$ have a wanted property. The
probability of $k$ elements having property $K$ when choosing $n$ is
\[
H = \frac{
	\left(\begin{array}{c}K\\k\end{array}\right)
	\cdot 
	\left(\begin{array}{c}N-K\\n-k\end{array}\right)
}
{
	\left(\begin{array}{c}N\\n\end{array}\right)
}
\]
\subsection{Newton Method}
%TODO(Alex): More information when to use.
The intersections of a function $f$ with the x-axis can be
approximated iteratively:
\[
x_{n + 1} = x_n - \frac{f(x_n)}{f'(x_n)} 
\]
The method converges towards the solution quadratically therefore
doubleing the amount of correct decimal places every iteration.


%%% Local Variables:
%%% mode: latex
%%% TeX-master: "../main"
%%% End:
