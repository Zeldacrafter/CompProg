\section{Number Theory}
\lst{GCD and LCM}{$\mathcal{O}(\log(a + b))$}{math/gcd_lcm.cc}
\begin{code}{Extended Euclid}{$\mathcal{O}(\log(a + b))$}{math/extendedEuclid.cc} 
 Calculates $a$ and $b$ such that $\gcd(x, y) = ax + by$
\end{code}

\begin{code}{Coefficients of a Polynomial}{$\mathcal{O}(n^2)$}{math/getCoefficiants.cc}
  For any polynomial with $p(x_i) = y_i$ where all $x_i$ are distinct coefficients can be calculated the following way:
\end{code}

\lst{Modular Exponentiation}{$\mathcal{O}(\log b)$}{math/modpow.cc}

\begin{code}{Miller Rabin Prime Test}{$\mathcal{O}(7\log b)$}{math/millerRabin.cc}
  Works for all numbers up to $2^{64}$
\end{code}

\begin{code}{Eulers Totoid Function}{$\mathcal{O}(\sqrt{n})$}{math/totoid.cc}
%TODO(Alex): Improve this -> better(more) details/ what is important, applications
$\phi(n)$: Number of Integers in $[1, n]$ that are coprime to $n$.\\
$p \text{ prime}, k \in \mathbb{N} \Rightarrow \phi(p^k) = p^k - p^{k-1}$ \\
$a, b \text{ coprime} \Rightarrow \phi(a \cdot b) = \phi(a) \cdot \phi(b)$ \\
$a, b \in \mathbb{N} \Rightarrow \phi(a \cdot b) = \phi(a) \cdot \phi(b) \cdot \frac{gcd(a, b)}{\phi(gcd(a,b))}$ \\
$P \text{ prime factors of } n \Rightarrow \phi(n) = n \cdot \prod_{p \in P} (1 - \frac{1}{p})$
\end{code}


%%% Local Variables:
%%% mode: latex
%%% TeX-master: "../main"
%%% End:
