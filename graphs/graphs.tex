\section{Graphs}
\subsection{Traversal}

\underline{Runtime:}\\
With adjacency list: $\mathcal{O}(\lvert V \rvert + \lvert E \rvert)$ \\
With adjacency matrix: $\mathcal{O}(\lvert V \rvert^2)$
\lst{DFS}{graphs/DFS.cc}
\underline{Runtime:}\\
With adjacency list: $\mathcal{O}(\lvert V \rvert + \lvert E \rvert)$ \\
With adjacency matrix: $\mathcal{O}(\lvert V \rvert^2)$
\lst{BFS}{graphs/BFS.cc}

\subsection{Topological sort}

\underline{Runtime:} $\mathcal{O}(\lvert V \rvert + \lvert E \rvert)$
\lst{DFS}{graphs/topoSortDFS.cc}
\underline{Runtime:} $\mathcal{O}(\lvert V \rvert + \lvert E \rvert)$
\lst{Kahn}{graphs/topoSortKahn.cc}

\subsection{Connected components}

\subsubsection{Bridges}

Bridges are edges that increase the number of connected components if
removed.

\underline{Runtime:} $\mathcal{O}(\lvert V \rvert + \lvert E \rvert)$
\lstwot{graphs/bridges.cc}

\subsubsection{Articulation Points}

Articulation Points are vertices that increase the number of connected
components if removed.

\underline{Runtime:} $\mathcal{O}(\lvert V \rvert + \lvert E \rvert)$
\lstwot{graphs/articulationPoints.cc}

\subsubsection{Strongly connected components}
\underline{Runtime:} $\mathcal{O}(\lvert V \rvert + \lvert E \rvert)$
\lst{SCC Tarjan}{graphs/scc.cc}


\subsection{Minimal spanning tree}
\underline{Runtime:} $\mathcal{O}(\lvert E \rvert \log \lvert V
\rvert) = \mathcal{O}(\lvert E \rvert \log \lvert E \rvert)$
\lst{Kruskal}{graphs/Kruskal.cc}

\subsection{Shortest paths}

\subsubsection{Single source shortest path}
\underline{Runtime:}
$\mathcal{O}((\lvert E \rvert + \lvert V \rvert )\log \lvert V
\rvert)$
\lst{Djikstra}{graphs/Djikstra.cc}

\underline{Runtime:} $\mathcal{O}(\lvert E \rvert \lvert V \rvert)$
\lst{Bellman Ford}{graphs/BellmanFord.cc}

This approach may be faster
\lst{Bellman Ford w/ Queue}{graphs/BellmanFordQueue.cc}
\subsubsection{All pairs shortest paths}
\underline{Runtime:} $\mathcal{O}(\lvert V \rvert^3)$
\lst{Floyd Warshall}{graphs/FloydWarshall.cc}

\subsection{Maximum flow}

\textbf{Max-flow min-cut theorem.} The maximum value of an $s$-$t$ flow is
equal to the minimum capacity over all $s$-$t$ cuts

\underline{Runtime:} $\mathcal{O}(\lvert V \rvert^5)$
\lst{Edmonds Karp w/ adjacency matrix}{graphs/EdmondsKarpMatrix.cc}

\underline{Runtime:} $\mathcal{O}(\lvert V \rvert \lvert E \rvert^2)$
\lst{Edmonds Karp w/ adjacency lists}{graphs/EdmondsKarpLists.cc}

\subsubsection{Minimum s-t cut}

To find a minimal $s$-$t$ cut find all nodes that are reachable in the
residual network for a network w/ maximum flow from $s$ and $t$
respectively.

\lstwot{graphs/s-t.cc}

%%% Local Variables:
%%% mode: latex
%%% TeX-master: "../main"
%%% End:
