\section{Graphs}

\begin{code}{Topological Sort}{$\mathcal{O}(|V| + |E|)$}{graphs/topoSort.cc}
  A priority queue can be used if further sorting is necessary.
\end{code}


\lst{SCC Tarjan}{$\mathcal{O}(|V| + |E|)$}{graphs/scc.cc}
\lst{Articulation Points}{$\mathcal{O}(| V | + | E |)$}{graphs/articulationPoints.cc}
\lst{Bridges}{$\mathcal{O}(|V|+|E|)$}{graphs/bridges.cc}


\lst{Minimal Spanning Tree -- Kruskal}{$\mathcal{O}(|E|\log|V|)$}{graphs/kruskal.cc}

\subsection{Shortest Paths}

\lst{Dijkstra}{$\mathcal((|E| + |V|)\log|V|)$}{graphs/dijkstra.cc}

\begin{code}{Bellman Ford}{$\mathcal{O}(|E||V|)$}{graphs/bellmanFord.cc}
	Check for negative cycles: \\
	$dist$ still changes in a $\lvert V \rvert$'th relaxation step with $dist_i = 0$ initially for all $i$.
\end{code}

\begin{code}{Bellman Ford with Queue}{$\mathcal{O}(|E||V|)$}{graphs/bellmanFordQueue.cc}
  This approach may be faster
\end{code}

\lst{Floyd Warshall}{$\mathcal{O}(|V|^3)$}{graphs/floydWarshall.cc}

\subsection{Forest}
\lst{Lowest Common Ancestor}{build: $\mathcal{O}(|V| \log|V|)$ query: $\mathcal{O}(1)$}{graphs/LCA.cc}
\lst{LCA with binary lifting}{build: $\mathcal{O}(|V|\log |V|)$ query: $\mathcal{O}(\log|V|)$}{graphs/LCABL.cc}

\lst{Heavy-light decomposition}{build: $\mathcal{O}(|V|)$, query/update: $\mathcal{O}(\log^2|V|)$/$\mathcal{O}(\log|V|)$}{graphs/HLD.cc}


\subsection{Flow}

\textbf{Max-flow min-cut theorem.} The maximum value of an $s$-$t$ flow is
equal to the minimum capacity over all $s$-$t$ cuts

\lst{Edges for flow algorithms}{}{graphs/flowedge.cc}
\lst{Edmonds Karp}{$\mathcal{O}(|V||E|^2)$}{graphs/edmondsKarp.cc}
\lst{Dinic}{$\mathcal{O}(|V|^2|E|)$}{graphs/dinic.cc}
\lst{Push Relabel}{$\mathcal{O}(|V|^3)$}{graphs/pushRelabel.cc}

\subsubsection{Minimum s-t cut}

To find a minimal $s$-$t$ cut find all nodes that are reachable in the
residual network for a network w/ maximum flow from $s$.  This is the
$s$ part of the cut.  All other nodes belong to the $t$ part.

\subsubsection{Closure Problem}

A closure of a directed graph is a set of vertices with no outgoing
edges.  The closure problem is the task to find the maximum weighted
closure.  Solvable through reduction to a maximum flow problem: Add
source and target, connect all the vertices with positive weight $w$
to the source with capacity $w$ and connect all the vertices with
negative weight $w$ to the target with capacity $-w$.  All of the
edges in the original graph have infinite capacity in the new
graph.  The weight of the maximum weighted closure is equal to the sum
of all positive weighted vertices in the original graph minus the
maximum flow in the constructed graph.

