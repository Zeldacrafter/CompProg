\documentclass{article}


\usepackage[a4paper,margin=2cm,bottom=2cm,top=2cm,landscape,twocolumn]{geometry}
\usepackage[english]{babel}
\usepackage{lmodern}
\usepackage[T1]{fontenc}       
\usepackage[utf8]{inputenc}    

% diagonal  seperator in table
\usepackage{diagbox}

% font for more readable code
\usepackage{dejavu}

\usepackage{xparse}

% math packages
\usepackage{amsmath}
\usepackage{amssymb}
\usepackage{amsthm}

\usepackage{tabu}

% spacing for section headings
\usepackage{titlesec} 
\titlespacing*{\section}{0pt}{0pt}{0pt}
\titlespacing*{\subsection}{0pt}{0pt}{0pt}
\titlespacing*{\subsubsection}{0pt}{0pt}{0pt}


% images
\usepackage{graphicx}

% spacing for itemize/enumerate
\usepackage{enumitem}
\setlist{nosep}


% more spacing related settings
\setlength\parindent{0pt}
\setlength\parskip{\baselineskip}

\setcounter{secnumdepth}{0}

% code listings
\usepackage[usenames,dvipsnames,svgnames,table]{xcolor}
\usepackage[skins, breakable]{tcolorbox}
\usepackage{fontawesome5}
\usepackage{wallpaper}

\usepackage{minted}
\setminted[c++]{style=perldoc, fontsize=\footnotesize}

\NewDocumentCommand\lst{mmm}{ %
  \tcbset{enhanced, colframe=black, arc=0pt, boxsep=0pt, left=2pt, right=2pt}
  \begin{tcolorbox}[ %
    arc=1pt,
    title=#1,
    colback=white,
    leftupper=0pt,
    rightupper=0pt,
    top=2pt,
    bottom=2pt,
    boxrule=1pt,
    drop fuzzy shadow,
    lower separated=false,
    toptitle=2pt,
    bottomtitle=2pt,
    breakable,
    break at=-\baselineskip/0pt,
    fonttitle=\bfseries\sffamily,
    before title={\small\faCode\hspace*{2pt}},
    after title={\hfill\normalfont\footnotesize #2}]
    \inputminted[firstline=2]{c++}{../code/#3}
  \end{tcolorbox}
}

\NewDocumentEnvironment{code}{mmm}{ %
  \tcbset{enhanced, colframe=black, arc=0pt, boxsep=0pt, left=2pt, right=2pt}
  \begin{tcolorbox}[ %
    arc=1pt,
    title=#1,
    colback=white,
    leftupper=0pt,
    rightupper=0pt,
    top=0pt,
    bottom=2pt,
    boxrule=1pt,
    middle=1pt,
    drop fuzzy shadow,
    lower separated=false,
    toptitle=2pt,
    bottomtitle=2pt,
    breakable,
    break at=-\baselineskip/0pt,
    fonttitle=\bfseries\sffamily,
    before title={\small\faCode\hspace*{2pt}},
    after title={\hfill\normalfont\footnotesize #2}]
    \begin{tcolorbox}[ %
      enhanced,
      arc=0pt,
      boxsep=0pt,
      left=2pt,
      right=2pt,
      bottom=2pt,
      top=2pt,
      boxrule=0pt,
      borderline south={1pt}{0pt}{black}]
      \begin{small}}{ %
      \end{small}
    \end{tcolorbox}
    \tcblower
    \inputminted[firstline=2]{c++}{../code/#3}
  \end{tcolorbox}
}


% header
\usepackage{fancyhdr}
\pagestyle{fancy}

\setlength{\headheight}{1.2cm}
\setlength{\headsep}{0.5cm}

\lhead{Kiel University}
\chead{}
\rhead{\thepage}

\rfoot[]{}
\lfoot[]{}
\cfoot[]{}

\usepackage{tocloft}
\setlength{\cftbeforesecskip}{6pt}
\usepackage[toc]{multitoc}
\renewcommand*{\multicolumntoc}{2}
\setlength{\columnseprule}{0.5pt}

\usepackage[en-GB]{datetime2}
\makeatletter
\newcommand{\daymonthyear}{%
  \@dtm@day. \DTMenglishmonthname{\@dtm@month} \@dtm@year
}
\def\blfootnote{\gdef\@thefnmark{}\@footnotetext}
\makeatother
% multi column layout
\usepackage{multicol}
\setlength{\columnseprule}{0.4pt}

\AtBeginDocument{\addtocontents{toc}{\protect\thispagestyle{fancy}}} 

\begin{document}
\begin{titlepage}
  \ThisCenterWallPaper{1.0}{assets/obligatoryoptimisticoctopus.pdf}
  \vspace*{44em}
  \hfill\daymonthyear
\end{titlepage}
      
  \setcounter{tocdepth}{1}
  \tableofcontents
  \blfootnote{Code icon by Font Awesome}
  \lst{Template}{}{template.cc}
  \lst{.vimrc}{}{utils/.vimrc}
  \section{Math}

\subsection{Geometric Series}
Geometric series for $q \neq 1$:
\[
\sum\limits_{k=1}^n q^k = \frac{1-q^{n+1}}{1-q}
\]
For $\lvert q \rvert < 1$ the corresponding infinite sum converges:
\[
\sum_{i = 1}^{\infty} q^k = \frac{1}{1 - q}
\]

\subsection{Picks Theorem}
For every polygon with integer-only coordinates with following holds for the area $A$, amount of interior integer points $i$ and amount of boundary points $b$:
\[
A = i + \frac{b}{2} - 1
\]
Calculation of border points:
\lstinputlisting{geometry/picksTheorem.cc}
For polygon area see the geometry section.

\subsection{Eulers Formular for Planar Graphs}
For every planar graph $G$ the following holds:
\[
V - E + F = 2
\]

\subsection{Combinatorics}
%TODO(Alex): Basic combinatorics
\subsubsection{Derangements}
Amount of permutations of a set with $n$ elements such that no element is at its starting position.
\begin{lstlisting} 
der(0) = 1
der(1) = 0
der(n) = (n-1)*(der(n-1)+der(n-2))
\end{lstlisting}
\subsubsection{Hypergeometric distribution}
Set with $N$ elements of which $K$ have a wanted property. The probability of $k$ elements having property $K$ when choosing $n$ is
\[
H = \frac{
	\left(\begin{array}{c}K\\k\end{array}\right)
	\cdot 
	\left(\begin{array}{c}N-K\\n-k\end{array}\right)
}
{
	\left(\begin{array}{c}N\\n\end{array}\right)
}
\]
\subsection{Newton Method}
%TODO(Alex): More information when to use.
The intersections of a function $f$ with the x-axis can be approximated iteratively:
\[
x_{n + 1} = x_n - \frac{f(x_n)}{f'(x_n)} 
\]
The method converges towards the solution quadratically therefore doubleing the amount of correct decimal places every iteration.
  
  \section{Number Theory}
\subsection{GCD and LCM}
\lstinputlisting{numberTheory/gcd_lcm.cc}

\subsection{Coefficients of a Polynomial}
\underline{Runtime:} $\mathcal{O}(n^2)$ \\
For any polynomial with $p(x_i) = y_i$ where all $x_i$ are distinct
coefficients can be calculated the following way:
\lstinputlisting{numberTheory/getCoefficiants.cc}
This works for any field. Not just $ints$/$reals$.

\subsection{Modular Exponentiation}
\underline{Runtime:} $\mathcal{O}(\log n)$
\lstinputlisting{numberTheory/modpow.cc}

\subsection{Probabilistic Prime Number Test}
\underline{Runtime:} $\mathcal{O}(iter \cdot \log^3 n)$
\lstinputlisting[title=MillerRabin]{numberTheory/millerRabin.cc}
$4^{-iter}$ chance for false positive ($1 - 4^{-10} \approx 99.999905\%$). If MillerRabin returns false the number is guaranteed composite.
%TODO(Alex): Deterministic version for 64 bit integers.
  \section{Data structures}

\subsection{Trees}

\underline{Runtime:}
\begin{itemize}
\item Build up: $\mathcal{O}(n\log n)$
\item query: $\mathcal{O}(\log n)$
\item (point) update: $\mathcal{O}(\log n)$
\end{itemize}
\lst{Fenwick tree}{dataStructures/fenwickTree.cc}


\underline{Runtime:}
\begin{itemize}
\item Build up: $\mathcal{O}(n)$
\item query: $\mathcal{O}(\log n)$
\item (point) update: $\mathcal{O}(\log n)$
\end{itemize}
\lst{Segment tree}{dataStructures/segmentTree.cc}

\subsection{Union find}
\underline{Runtime:}\\
For a set with $n$ elements $m \geq n$ mixed union and find operations need a maximum of $\mathcal{O}(m \cdot \alpha(m, n)) \approx \mathcal{O}(n)$ time.
\lstwot{dataStructures/unionFind.cc}

\subsection{Sparse Table}
Range queries for any function $f$ where reapplication does not change the result (min, max, gcd, lcm). Dynamic updates are not possible.
%TODO(Alex): Other functions (sum..)
\underline{Runtime:}
\begin{itemize}
	\item Build: $\mathcal{O}(n \log n)$
	\item Query: $\mathcal{O}(1)$
\end{itemize}
\lst{Sparse Table}{dataStructures/sparseTable.cc}

%%% Local Variables:
%%% mode: latex
%%% TeX-master: "../main"
%%% End:

  \section{Dynamic Programming}
\lst{Knapsack}{$\mathcal{O}(n\sum_{i = 1}^n p_i)$}{dynamicProgramming/knapsack.cc}
\lst{TSP}{$\mathcal(n * 2^n)$}{dynamicProgramming/tsp.cc}
\lst{Subset Sum}{$\mathcal{O}(n\sum_{i = 1}^n v_i)$}{dynamicProgramming/subSetSum.cc}
\lst{Edit Distance}{$\mathcal{O}(nm)$}{dynamicProgramming/editDistance.cc}
\lst{Longest Increasing Subsequence}{$\mathcal{O}(n\log n)$}{dynamicProgramming/lis.cc}
\lst{Longest Common Subsequence}{$\mathcal{O}(nm)$}{dynamicProgramming/lcs.cc}

%%% Local Variables:
%%% mode: latex
%%% TeX-master: "../main"
%%% End:

  \section{Graphs}

\subsection{Topological sort}

\underline{Runtime:} $\mathcal{O}(\lvert V \rvert + \lvert E \rvert)$
\lst{Kahn}{graphs/topoSort.cc}

\subsection{Connected components}

\subsubsection{Bridges}

Bridges are edges that increase the number of connected components if
removed.

\underline{Runtime:} $\mathcal{O}(\lvert V \rvert + \lvert E \rvert)$
\lstwot{graphs/bridges.cc}

\subsubsection{Articulation Points}

Articulation Points are vertices that increase the number of connected
components if removed.

\underline{Runtime:} $\mathcal{O}(\lvert V \rvert + \lvert E \rvert)$
\lstwot{graphs/articulationPoints.cc}

\subsubsection{Strongly connected components}
\underline{Runtime:} $\mathcal{O}(\lvert V \rvert + \lvert E \rvert)$
\lst{SCC Tarjan}{graphs/scc.cc}


\subsection{Minimal spanning tree}
\underline{Runtime:} $\mathcal{O}(\lvert E \rvert \log \lvert V
\rvert) = \mathcal{O}(\lvert E \rvert \log \lvert E \rvert)$
\lst{Kruskal}{graphs/kruskal.cc}

\subsection{Shortest paths}

\subsubsection{Single source shortest path}
\underline{Runtime:}
$\mathcal{O}((\lvert E \rvert + \lvert V \rvert )\log \lvert V
\rvert)$
\lst{Djikstra}{graphs/djikstra.cc}

\underline{Runtime:} $\mathcal{O}(\lvert E \rvert \lvert V \rvert)$
\lst{Bellman Ford}{graphs/bellmanFord.cc}

This approach may be faster
\lst{Bellman Ford w/ Queue}{graphs/bellmanFordQueue.cc}
\subsubsection{All pairs shortest paths}
\underline{Runtime:} $\mathcal{O}(\lvert V \rvert^3)$
\lst{Floyd Warshall}{graphs/floydWarshall.cc}

\subsubsection{Lowest common ancestor}
\underline{Runtime:} build: $\mathcal{O}(\lvert V \rvert \log \lvert
V\rvert)$ query: $\mathcal{O}(1)$
\lstwot{graphs/lca.cc}

\subsection{Maximum flow}

\textbf{Max-flow min-cut theorem.} The maximum value of an $s$-$t$ flow is
equal to the minimum capacity over all $s$-$t$ cuts

\lst{Edges for flow algorithms}{graphs/flowedge.cc}
\underline{Runtime:} $\mathcal{O}(\lvert V \rvert \lvert E \rvert^2)$
\lst{Edmonds Karp}{graphs/edmondsKarp.cc}

\underline{Runtime:} $\mathcal{O}(\left| V \right|^3)$
\lst{Push Relabel}{graphs/pushRelabel.cc}

\subsubsection{Minimum s-t cut}

To find a minimal $s$-$t$ cut find all nodes that are reachable in the
residual network for a network w/ maximum flow from $s$ and $t$
respectively.

\lstwot{graphs/s-t.cc}

%%% Local Variables:
%%% mode: latex
%%% TeX-master: "../main"
%%% End:

  \section{Strings}
\lst{Trie/Prefix Tree}{$\mathcal{O}(n)$ for insert and get}{strings/trie.cc}

\subsection{Prefix Function}
For a string $s$ return an array in which the $i$-th entry is the
length of the longest proper prefix of $s[0,\ldots, i]$ which is also
a suffix.  Note that the $0$-th entry is $0$

\lst{Prefix Function}{$\mathcal(O)(n)$}{strings/prefixFunction.cc}

\subsection{KMP}
Returns a list with all the starting indeces where the pattern matches
the text.

\lst{KMP}{$\mathcal{O}(n + m)$}{strings/kmp.cc}

\subsection{Manachers}
% source: https://www.hackerrank.com/topics/manachers-algorithm
Returns array where $P[i]$ contains the length of the palindrome with
mid-point $i$. There are extra entries for where the mid-point is
inbetween letters. Example for $abbaac$:
\begin{lstlisting}
| a | b | b | a | a | a | c |
0 1 0 1 4 1 0 1 2 3 2 1 0 1 0
\end{lstlisting}

\lst{Manachers}{$\mathcal{O}(n)$}{strings/manachers.cc}
\subsection{Aho-Corasick Automaton}
% See for detailed explaination:
% http://web.stanford.edu/class/archive/cs/cs166/cs166.1166/lectures/02/Small02.pdf
% TODO(Alex): Test query() call extensively. Build + insert are okay for sure.
Data structure to search how often $n$ fixed strings $t_1, \dots, t_n$
are contained in a variable string $S$. Strings $t_j$ may each have
values.

\lst{Aho-Corasick Automaton}{build: $\mathcal{O}(\sum |t_j|)$, query: $\mathcal{O}(|S|)$}{strings/aho_corasick.cc}

%%% Local Variables:
%%% mode: latex
%%% TeX-master: "../main"
%%% End:
  \section{Geometry}
\lst{Geometry}{}{geometry/geometry.cc}

\lst{Polygon}{inPolygon: $\mathcal{O}(\log n)$, area: $\mathcal{O}(n)$, isConvex: $\mathcal{O}(n)$}{geometry/polygon.cc}

\lst{Convex Hull}{$\mathcal{O}(n\log n)$}{geometry/convexHull.cc}


  \section{Utils}

\lst{Fast IO}{}{utils/fastio.cc}

%TODO(Alex): Make output routine better
\lst{128 Bit Integer}{}{utils/128BitInt.cc}

\lst{Buildin}{}{utils/buildin.cc}

\lst{Bit Operations}{}{utils/bits.cc}

\lst{Ternary Search}{$\mathcal{O}(\log (r - l)$}{utils/ternarySearch.cc}


%%% Local Variables:
%%% mode: latex
%%% TeX-master: "../main.tex"
%%% End:


\end{document}

%%% Local Variables:
%%% TeX-command-extra-options: "-shell-escape"
%%% mode: latex
%%% TeX-master: t
%%% End:
