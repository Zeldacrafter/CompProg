\documentclass{article}


\usepackage[a4paper,margin=2cm,bottom=2cm,top=2cm,landscape,twocolumn]{geometry}
\usepackage[english]{babel}
\usepackage{lmodern}
\usepackage[T1]{fontenc}       
\usepackage[utf8]{inputenc}    

% diagonal  seperator in table
\usepackage{diagbox}

% font for more readable code
\usepackage{dejavu}

\usepackage{xparse}

% math packages
\usepackage{amsmath}
\usepackage{amssymb}
\usepackage{amsthm}

\usepackage{tabu}

% spacing for section headings
\usepackage{titlesec} 
\titlespacing*{\section}{0pt}{0pt}{0pt}
\titlespacing*{\subsection}{0pt}{0pt}{0pt}
\titlespacing*{\subsubsection}{0pt}{0pt}{0pt}


% images
\usepackage{graphicx}

% spacing for itemize/enumerate
\usepackage{enumitem}
\setlist{nosep}


% more spacing related settings
\setlength\parindent{0pt}
\setlength\parskip{\baselineskip}

\setcounter{secnumdepth}{0}

% code listings
\usepackage[usenames,dvipsnames,svgnames,table]{xcolor}
\usepackage[skins, breakable]{tcolorbox}
\usepackage{fontawesome5}
\usepackage{wallpaper}

\usepackage{minted}
\setminted[c++]{style=perldoc, fontsize=\footnotesize}

\NewDocumentCommand\lst{mmm}{ %
  \tcbset{enhanced, colframe=black, arc=0pt, boxsep=0pt, left=2pt, right=2pt}
  \begin{tcolorbox}[ %
    arc=1pt,
    title=#1,
    colback=white,
    leftupper=0pt,
    rightupper=0pt,
    top=2pt,
    bottom=2pt,
    boxrule=1pt,
    drop fuzzy shadow,
    lower separated=false,
    toptitle=2pt,
    bottomtitle=2pt,
    breakable,
    break at=-\baselineskip/0pt,
    fonttitle=\bfseries\sffamily,
    before title={\small\faCode\hspace*{2pt}},
    after title={\hfill\normalfont\footnotesize #2}]
    \inputminted[firstline=2]{c++}{../code/#3}
  \end{tcolorbox}
}

\NewDocumentEnvironment{code}{mmm}{ %
  \tcbset{enhanced, colframe=black, arc=0pt, boxsep=0pt, left=2pt, right=2pt}
  \begin{tcolorbox}[ %
    arc=1pt,
    title=#1,
    colback=white,
    leftupper=0pt,
    rightupper=0pt,
    top=0pt,
    bottom=2pt,
    boxrule=1pt,
    middle=1pt,
    drop fuzzy shadow,
    lower separated=false,
    toptitle=2pt,
    bottomtitle=2pt,
    breakable,
    break at=-\baselineskip/0pt,
    fonttitle=\bfseries\sffamily,
    before title={\small\faCode\hspace*{2pt}},
    after title={\hfill\normalfont\footnotesize #2}]
    \begin{tcolorbox}[ %
      enhanced,
      arc=0pt,
      boxsep=0pt,
      left=2pt,
      right=2pt,
      bottom=2pt,
      top=2pt,
      boxrule=0pt,
      borderline south={1pt}{0pt}{black}]
      \begin{small}}{ %
      \end{small}
    \end{tcolorbox}
    \tcblower
    \inputminted[firstline=2]{c++}{../code/#3}
  \end{tcolorbox}
}


% header
\usepackage{fancyhdr}
\pagestyle{fancy}

\setlength{\headheight}{1.2cm}
\setlength{\headsep}{0.5cm}

\lhead{Kiel University}
\chead{}
\rhead{\thepage}

\rfoot[]{}
\lfoot[]{}
\cfoot[]{}

\usepackage{tocloft}
\setlength{\cftbeforesecskip}{6pt}
\usepackage[toc]{multitoc}
\renewcommand*{\multicolumntoc}{2}
\setlength{\columnseprule}{0.5pt}

\usepackage[en-GB]{datetime2}
\makeatletter
\newcommand{\daymonthyear}{%
  \@dtm@day. \DTMenglishmonthname{\@dtm@month} \@dtm@year
}
\def\blfootnote{\gdef\@thefnmark{}\@footnotetext}
\makeatother
% multi column layout
\usepackage{multicol}
\setlength{\columnseprule}{0.4pt}

\AtBeginDocument{\addtocontents{toc}{\protect\thispagestyle{fancy}}} 

\begin{document}
\begin{titlepage}
  \ThisCenterWallPaper{1.0}{assets/obligatoryoptimisticoctopus.pdf}
  \vspace*{44em}
  \hfill\daymonthyear
\end{titlepage}
      
  \setcounter{tocdepth}{1}
  \tableofcontents
  \blfootnote{Code icon by Font Awesome}
  \lst{Template}{}{template.cc}
  \lst{.vimrc}{}{utils/.vimrc}
  \section{Math}

\subsection{Geometric Series}
Geometric series for $q \neq 1$:
\[
\sum\limits_{k=1}^n q^k = \frac{1-q^{n+1}}{1-q}
\]
For $\lvert q \rvert < 1$ the corresponding infinite sum converges:
\[
\sum_{i = 1}^{\infty} q^k = \frac{1}{1 - q}
\]

\subsection{Derangements}
Amount of permutations of a set with $n$ elements such that no element is at its starting position.
\begin{lstlisting} 
der(0) = 1
der(1) = 0
der(n) = (n-1)*(der(n-1)+der(n-2))
\end{lstlisting}

\subsection{Picks Theorem}
For every polygon with integer-only coordinates with following holds for the area $A$, amount of interior integer points $i$ and amount of boundary points $b$:
\[
A = i + \frac{b}{2} - 1
\]
Calculation of border points:
\lstinputlisting{geometry/picksTheorem.cc}
For polygon area see the geometry section.

\subsection{Eulers Formular for Planar Graphs}
For every planar graph $G$ the following holds:
\[
V - E + F = 2
\]  
  \section{Number Theory}
\lst{GCD and LCM}{$\mathcal{O}(\log(a + b))$}{math/gcd_lcm.cc}
\lst{extended euclid}{$\mathcal{O}(\log(a +
  b))$}{math/extendedEuclid.cc} 


\lst[For any polynomial with $p(x_i) = y_i$ where all $x_i$ are distinct coefficients can be calculated the following way:]{Coefficients of a Polynomial}{$\mathcal{O}(n^2)$}{math/getCoefficiants.cc}
\subsection{Modular Exponentiation}
\lst{binary exponentiation}{$\mathcal{O}(\log b)$}{math/modpow.cc}

\subsection{Miller Rabin Prime Test}
Works for all numbers up to $2^{64}$.

\lst{Miller Rabin}{$\mathcal{O}(7\log b)$}{math/millerRabin.cc}

\subsection{Eulers Totoid Function}
%TODO(Alex): Improve this -> better(more) details/ what is important, applications
$\phi(n)$: Number of Integers in $[1, n]$ that are coprime to $n$.\\
$p \text{ prime}, k \in \mathbb{N} \Rightarrow \phi(p^k) = p^k - p^{k-1}$ \\
$a, b \text{ coprime} \Rightarrow \phi(a \cdot b) = \phi(a) \cdot \phi(b)$ \\
$a, b \in \mathbb{N} \Rightarrow \phi(a \cdot b) = \phi(a) \cdot \phi(b) \cdot \frac{gcd(a, b)}{\phi(gcd(a,b))}$ \\
$P \text{ prime factors of } n \Rightarrow \phi(n) = n \cdot \prod_{p \in P} (1 - \frac{1}{p})$

\lst{totoid}{$\mathcal{O}(\sqrt{n})$}{math/totoid.cc}


%%% Local Variables:
%%% mode: latex
%%% TeX-master: "../main"
%%% End:

  \section{Data structures}

\subsection{Trees}

\underline{Runtime:}
\begin{itemize}
\item Build up: $\mathcal{O}(n\log n)$
\item query: $\mathcal{O}(\log n)$
\item (point) update: $\mathcal{O}(\log n)$
\end{itemize}
\lst{Fenwick tree}{dataStructures/FT.cc}


\underline{Runtime:}
\begin{itemize}
\item Build up: $\mathcal{O}(n)$
\item query: $\mathcal{O}(\log n)$
\item (point) update: $\mathcal{O}(\log n)$
\end{itemize}
\lst{Segment tree}{dataStructures/ST.cc}

\subsection{Union find}
\underline{Runtime:}\\
For a set with $n$ elements $m \geq n$ mixed union and find operations need a maximum of $\mathcal{O}(m \cdot \alpha(m, n)) \approx \mathcal{O}(n)$ time.
\lstwot{dataStructures/DSU.cc}

\subsection{Sparse Table}
Range queries for any function $f$ where reapplication does not change the result (min, max, gcd, lcm). Dynamic updates are not possible.
%TODO(Alex): Other functions (sum..)
\underline{Runtime:}
\begin{itemize}
\item Build: $\mathcal{O}(n \log n)$
\item Query: $\mathcal{O}(1)$
\end{itemize}
\lst{1D Sparse Table}{dataStructures/SPT.cc}
\underline{Runtime:}
\begin{itemize}
\item Build: $\mathcal{O}(n\cdot m \cdot \log n \cdot \log m)$
\item Query: $\mathcal{O}(1)$
\end{itemize}
%TODO(Alex): Formatting in code
\lst{2D Sparse Table}{dataStructures/sparseTable2D.cc}
%%% Local Variables:
%%% mode: latex
%%% TeX-master: "../main"
%%% End:

  \section{Dynamic Programming}
\lst{Knapsack}{$\mathcal{O}(n\sum_{i = 1}^n p_i)$}{dynamicProgramming/knapsack.cc}
\lst{TSP}{$\mathcal(n * 2^n)$}{dynamicProgramming/tsp.cc}
\lst{Subset Sum}{$\mathcal{O}(n\sum_{i = 1}^n v_i)$}{dynamicProgramming/subSetSum.cc}
\lst{Edit Distance}{$\mathcal{O}(nm)$}{dynamicProgramming/editDistance.cc}
\lst{Longest Increasing Subsequence}{$\mathcal{O}(n\log n)$}{dynamicProgramming/lis.cc}
\lst{Longest Common Subsequence}{$\mathcal{O}(nm)$}{dynamicProgramming/lcs.cc}

%%% Local Variables:
%%% mode: latex
%%% TeX-master: "../main"
%%% End:

  \section{graphs}
\subsection{traversal}

\lstinputlisting[title=DFS]{graphs/DFS.cc}
\lstinputlisting[title=BFS]{graphs/BFS.cc}

\subsection{topological sort}

\lstinputlisting[title=DFS]{graphs/topoSortDFS.cc}
\lstinputlisting[title=Kahn]{graphs/topoSortKahn.cc}

\subsection{connected components}

\subsubsection{strongly connected components}
\lstinputlisting[title=SCC tarjan]{graphs/scc.cc}


\subsection{shortest paths}

\subsubsection{single source shortest path}



\subsubsection{all pairs shortest paths}

\lstinputlisting[title=Floyd Warshall]{graphs/FloydWarshall.cc}


%%% Local Variables:
%%% mode: latex
%%% TeX-master: "../main"
%%% End:

  \section{Strings}
\subsection{Manachers}
% source: https://www.hackerrank.com/topics/manachers-algorithm
Returns array where $P[i]$ contains the length of the palindrome with mid-point $i$.\\
There are extra entries for where the mid-point is inbetween letters.\\
Example for $abbaac$:
\begin{lstlisting}
| a | b | b | a | a | a | c |
0 1 0 1 4 1 0 1 2 3 2 1 0 1 0
\end{lstlisting}
\underline{Runtime:} $\mathcal{O}(n)$
\lst{Manachers}{strings/manachers.cc}
  \section{Geometry}
\subsection{Vector operations}
\lstinputlisting{geometry/geometry.cc}

\subsection{Polygon}
\lstinputlisting{geometry/polygon.cc}
\subsubsection{Convex Hull}
\underline{Runtime:} $\mathcal{O}(n \log n)$
\lstinputlisting{geometry/convexHull.cc}
  \section{Utils}

\subsection{fast I/O}
\lstwot{utils/fastio.cc}

\subsection{128 Bit Integer}
%TODO(Alex): Make output routine better
\lstwot{utils/128BitInt.cc}

\subsection{Buildins}
\lstwot{utils/buildin.cc}

\subsection{Bit Operations}
\lstwot{utils/bits.cc}
%%% Local Variables:
%%% mode: latex
%%% TeX-master: "../main.tex"
%%% End:


\end{document}

%%% Local Variables:
%%% TeX-command-extra-options: "-shell-escape"
%%% mode: latex
%%% TeX-master: t
%%% End:
