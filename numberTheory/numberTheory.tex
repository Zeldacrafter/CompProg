\section{Number Theory}
\subsection{GCD and LCM}
\lstwot{numberTheory/gcd_lcm.cc}

\subsection{Coefficients of a Polynomial}
\underline{Runtime:} $\mathcal{O}(n^2)$ \\
For any polynomial with $p(x_i) = y_i$ where all $x_i$ are distinct
coefficients can be calculated the following way:
\lstwot{numberTheory/getCoefficiants.cc}
This works for any field. Not just $ints$/$reals$.
\subsection{Modular Exponentiation}
\underline{Runtime:} $\mathcal{O}(\log n)$
\lstwot{numberTheory/modpow.cc}

\subsection{Probabilistic Prime Number Test}
\underline{Runtime:} $\mathcal{O}(iter \cdot \log^3 n)$
\lst{MillerRabin}{numberTheory/millerRabin.cc}
$4^{-iter}$ chance for false positive ($1 - 4^{-10} \approx 99.999905\%$). If MillerRabin returns false the number is guaranteed composite.
%TODO(Alex): Deterministic version for 64 bit integers.
%%% Local Variables:
%%% mode: latex
%%% TeX-master: "../main"
%%% End:
