\section{Number Theory}
\subsection{GCD and LCM}
\lstwot{numberTheory/gcd_lcm.cc}

\subsection{Coefficients of a Polynomial}
\underline{Runtime:} $\mathcal{O}(n^2)$ \\
For any polynomial with $p(x_i) = y_i$ where all $x_i$ are distinct
coefficients can be calculated the following way:
\lstwot{numberTheory/getCoefficiants.cc}
This works for any field. Not just $ints$/$reals$.
\subsection{Modular Exponentiation}
\underline{Runtime:} $\mathcal{O}(\log n)$
\lstwot{numberTheory/modpow.cc}

\subsection{Probabilistic Prime Number Test}
\underline{Runtime:} $\mathcal{O}(iter \cdot \log^3 n)$
\lst{MillerRabin}{numberTheory/millerRabin.cc}
$4^{-iter}$ chance for false positive ($1 - 4^{-10} \approx 99.999905\%$). If MillerRabin returns false the number is guaranteed composite. \\
There are no numbers that will always produce a false positive.
%TODO(Alex): Deterministic version for 64 bit integers.

\subsection{Eulers Totoid Function}
%TODO(Alex): Improve this -> better(more) details/ what is important, applications
$\phi(n)$: Number of Integers in $[1, n]$ that are coprime to $n$.\\
$p \text{ prime}, k \in \mathbb{N} \Rightarrow \phi(p^k) = p^k - p^{k-1}$ \\
$a, b \text{ coprime} \Rightarrow \phi(a \cdot b) = \phi(a) \cdot \phi(b)$ \\
$a, b \in \mathbb{N} \Rightarrow \phi(a \cdot b) = \phi(a) \cdot \phi(b) \cdot \frac{gcd(a, b)}{\phi(gcd(a,b))}$ \\
$P \text{ prime factors of } n \Rightarrow \phi(n) = n \cdot \prod_{p \in P} (1 - \frac{1}{p})$
\underline{Runtime:} $\mathcal{O}(\sqrt{n})$
\lstwot{numberTheory/totoid.cc}
